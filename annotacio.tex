\documentclass[12pt, a4paper, oneside]{book}
\usepackage[left=3.0cm, top=3.0cm, right=2.5cm, bottom=3.0cm]{geometry}
\usepackage[utf8]{inputenc}
\usepackage[T1]{fontenc}
\usepackage[english,hungarian]{babel}
\selectlanguage{hungarian}
\usepackage{fancyhdr}
\usepackage{setspace}

\pagestyle{fancy}

\begin{document}
\fancyhead[LR]{}
\titlepage
\begin{center}
\Large{Erlang/OTP}\\
\vspace{0.3cm}
\large{Magas rendelkezésre állású, elosztott rendszerek
fejlesztése}\\
\vspace{0.5cm}
\emph{annotáció}\\
\vspace{0.5cm}
{\small Készítette: Czinkos Zsolt, Gazdaságinformatikus Szak (BSc)}\\
\vspace{0.2cm}
{\small Konzulens: Dr. Fodor Szabina}\\
\vspace{0.5cm}
{\small 2012}
\vspace{1.5cm}
\end{center}

\onehalfspacing

Az Erlang nyelvet az Ericsson-nál hozták létre rendkívül magas rendelkezésre
állású telefonrendszerek fejlesztéséhez 1986-tól kezdődően. Az 1998-ban nyílt
forráskódúvá vált Erlang nyelvet -- és a platformba foglalt Open Telecom
Platform (OTP) keretrendszert -- ma már a telekom iparon kívül is használják,
elsősorban skálázható, magas rendelkezésre állású rendszerek építéséhez. Az
Erlang megalkotásánál az elsődleges cél magas rendelkezésre állású, hibatűrő
rendszerek építése volt (a folyamatos működést nem szakíthatja meg
szoftverfrissítés sem). A megfelelő redundancia csak egynél több géppel
biztosítható, ezért olyan megoldásra volt szükség, amely lehetővé teszi, hogy
magas szintű, erre a célra kialakított nyelven lehessen a megszokottnál
egyszerűbben párhuzamos (konkurens) programokat fejleszteni. Az Erlang két
alapra építve éri ezt el: 

\begin{itemize}
\item Egyrészt funkcionális nyelv, nincs megosztott állapot-változó a
rendszerben -- \emph{no shared state}, minden függvény megkap paraméterként
minden adatot, ami a feladata elvégzéséhez szükséges. Ugyanazokkal a
paraméterekkel meghívva mindig ugyanazt az eredményt adja vissza, akárcsak
például az \texttt{f(x) = x + 1} függvény (\emph{referential transaparency}).
\item Másrészt az Erlang folyamatok (\emph{process}) közti kommunikációt
aszinkron üzenetküldés teszi lehetővé; az aktor modellnek megfelelően
a rendszert aktorok alkotják, amelyek bizonyos viselkedésmintával rendelkeznek,
és képesek üzenetet küldeni, fogadni, további aktorokat létrehozni.
\end{itemize}

A dolgozat bemutatja az Erlang nyelv alapvető elemeit (példákkal segítve a
megértést) és az Open Telecom Platform szoftverkönyvtár alapelveit, működési
mechanizmusát. Az alapozás után egy egyszerű webes alkalmazás megépítését
követi végig lépésről lépésre, kódrészletekkel bemutatva a gyakorlatot. A
dolgozat végére összeálló rendszer tőzsdei kereskedési adatokat továbbít a
böngésző előtt ülő felhasználóknak, azonnal (\emph{real-time}) vagy a
beállított késleltetéssel. A szolgáltatást Erlang node-okból álló klaszter
biztosítja, amely a terhelés növekedésével egyszerűen, pusztán egy újabb node
hozzákapcsolásával bővíthető.

A példaprogram felépítése során a dolgozat bemutatja az Erlang azon
tulajdonságait, amelyek lehetővé teszik a folyamatos működést biztosító
rendszerek fejlesztését.

\end{document}
